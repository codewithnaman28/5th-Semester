\documentclass{article}
\usepackage[utf8]{inputenc}
\usepackage[margin=1in]{geometry}
\usepackage{listings}
\usepackage{xcolor}
\usepackage{booktabs}

\definecolor{codegreen}{rgb}{0,0.6,0}
\definecolor{codegray}{rgb}{0.5,0.5,0.5}
\definecolor{codepurple}{rgb}{0.58,0,0.82}
\definecolor{backcolour}{rgb}{0.95,0.95,0.92}

\lstdefinestyle{mystyle}
{
    backgroundcolor=\color{backcolour},   
    commentstyle=\color{codegreen},
    keywordstyle=\color{magenta},
    numberstyle=\tiny\color{codegray},
    stringstyle=\color{codepurple},
    basicstyle=\ttfamily\footnotesize,
    breakatwhitespace=false,         
    breaklines=true,                 
    captionpos=b,                    
    keepspaces=true,                 
    numbers=left,                    
    numbersep=5pt,                  
    showspaces=false,                
    showstringspaces=false,
    showtabs=false,                  
    tabsize=2
}

\lstset{style=mystyle}
\begin{document}
\begin{titlepage} % Suppresses displaying the page number on the title page and the subsequent page counts as page 1

	\raggedleft\rule{1pt}{\textheight} % Vertical line
	\hspace{0.05\textwidth} % Whitespace between the vertical line and title page text
	\parbox[b]{0.75\textwidth}
    { % Paragraph box for holding the title page text, adjust the width to move the title page left or right on the page
		
		{\Huge\bfseries MIT World Peace University \\[0.5\baselineskip] \ AIMLT}\\[2\baselineskip] % Title
		{\large\textit{Assignment 2}}\\[4\baselineskip] % Subtitle or further description
		{\Large\textsc{Naman Soni Roll No. 06}} % Author name, lower case for consistent small caps
		
		\vspace{0.5\textheight} % Whitespace between the title block and the publisher
	}

\end{titlepage}
\tableofcontents
\pagebreak
\section{\textbf{Title}}
Study of prolog programming language
\section{\textbf{Aim:}}
Demonstrate Reasoning/ Iaferencing using prolog
\section{\textbf{Theory}}
\subsection{\textit{What is Prolog?}}
Prolog is a short for programming logic is a programming language used in creating artigicial intelligence. Prolog is classified as a logic programming language and relies on the user to specify the rules and facts about a situation along with the end goal, otherwise known as a query
\subsection{\textit{What is symbolic language?}}
In prolog a symbolic language is form of declarative programming language that is particularly well-suited for representing and manipulating symbotic information and logical relationships. Some key characterstics of symbolic language in prolog are as follows:
\begin{itemize}
    \item \textbf{Declarative Syntax:} Prolog programs are written in a declarative syntax that emphasizes what should be done rather than hwo to do it. This makes it suitable for expressing symbolic relationships and logical rules.
    \item \textbf{Facts and Rules:} In prolog, you can define facts and rules that describe relationships between symbols and entities. Facts and Statements about the world and rules are used to infer new facts from existing ones using logical reasoning.
    \item \textbf{Predicate and Clauses:} Prolog programs consist of predicates and cluses. Predicates are used to define relationships adn clauses are composed of facts and rules. Predicates and Clauses are the building blocks of prolog programs.
\end{itemize}
\subsection{\textbf{Explain Rules, Facts and Queries}}
\subsubsection*{\textbf{Rules}}

\end{document}