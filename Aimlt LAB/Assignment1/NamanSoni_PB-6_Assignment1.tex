\documentclass{article}
\usepackage[utf8]{inputenc}
\usepackage[margin=1in]{geometry}
\usepackage{listings}
\usepackage{xcolor}
\usepackage{booktabs}

\definecolor{codegreen}{rgb}{0,0.6,0}
\definecolor{codegray}{rgb}{0.5,0.5,0.5}
\definecolor{codepurple}{rgb}{0.58,0,0.82}
\definecolor{backcolour}{rgb}{0.95,0.95,0.92}

\lstdefinestyle{mystyle}
{
    backgroundcolor=\color{backcolour},   
    commentstyle=\color{codegreen},
    keywordstyle=\color{magenta},
    numberstyle=\tiny\color{codegray},
    stringstyle=\color{codepurple},
    basicstyle=\ttfamily\footnotesize,
    breakatwhitespace=false,         
    breaklines=true,                 
    captionpos=b,                    
    keepspaces=true,                 
    numbers=left,                    
    numbersep=5pt,                  
    showspaces=false,                
    showstringspaces=false,
    showtabs=false,                  
    tabsize=2
}

\lstset{style=mystyle}
\begin{document}
\begin{titlepage} % Suppresses displaying the page number on the title page and the subsequent page counts as page 1

	\raggedleft\rule{1pt}{\textheight} % Vertical line
	\hspace{0.05\textwidth} % Whitespace between the vertical line and title page text
	\parbox[b]{0.75\textwidth}
    { % Paragraph box for holding the title page text, adjust the width to move the title page left or right on the page
		
		{\Huge\bfseries MIT World Peace University \\[0.5\baselineskip] \ AIMLT}\\[2\baselineskip] % Title
		{\large\textit{Assignment 2}}\\[4\baselineskip] % Subtitle or further description
		{\Large\textsc{Naman Soni Roll No. 06}} % Author name, lower case for consistent small caps
		
		\vspace{0.5\textheight} % Whitespace between the title block and the publisher
	}

\end{titlepage}
\tableofcontents
\pagebreak
\section{\textbf{Aim}}
Study State Space representation for AI problem solving.
\section{\textbf{Theory}}
\subsection{\textbf{\textit{State Space Search}}}
\textbf{State space search} is a problem-solving technique used in Artificial Intelligence (AI) to find the solution
path from the initial state to the goal state by exploring the various states. The state space search approach
searches through all possible states of a problem to find a solution. It is an essential part of Artificial
Intelligence and is used in various applications, from game-playing algorithms to natural language processing.

A state space is a way to mathematically represent a problem by defining all the possible states in which
the problem can be. This is used in search algorithms to represent the initial state, goal state, and current
state of the problem. Each state in the state space is represented using a set of variables.

The efficiency of the search algorithm greatly depends on the size of the state space, and it is important
to choose an appropriate representation and search strategy to search the state space efficiently.

One of the most well-known \textbf{state space search algorithms} is the A algorithm. Other commonly
used state space search algorithms include \textbf{breadth-first search (BFS), depth-first search (DFS), hill
climbing, simulated annealing, and genetic algorithms.}
\subsection{\textbf{\textit{Features of State Space Search}}}
\textbf{State Space Search} has several features that make it an effective problem-solving technique in Artificial
Intelligence. These features include:
\begin{itemize}
    \item \textbf{Exhaustiveness:} State space search explores all possible states of a problem to find a solution.
    \item \textbf{Completeness:} If a solution exists, state space search will find it.
    \item \textbf{Optimality:} Searching through a state space results in an optimal solution.
    \item \textbf{Uninformed and Informed Search:} State space search in artificial intelligence can be classified as
    uninformed if it provides additional information about the problem.
\end{itemize}

\end{document}